\documentclass[a4paper,9pt,fleqn]{article}

\usepackage{mathtools}
\usepackage{tabularx}
\usepackage{enumerate}
\usepackage[retainorgcmds]{IEEEtrantools}
\usepackage{tikz}
	\usetikzlibrary{automata, shapes, snakes}
\usepackage{pgf}
\usepackage{hyperref}
\setlength{\parindent}{0mm}
\setlength{\itemsep}{0ex}
\setlength{\parskip}{2ex}
\setlength{\parsep}{2ex}
\setlength{\partopsep}{0ex}
\setlength{\topsep}{0ex}



\title{
{\bf Report}\\
\vspace{0.2cm}
%
Home Assignment, DST15\_035\\
\vspace{1cm}
%
{\large Dennis Bennhage \& Hampus Lidin}\\
\vspace{10cm}
%
May 14th, 2015
}

\date{}

\begin{document}

\maketitle

\newpage

\section{System Overview}
The system we have constructed is intended to perform the first step in {\it Gauss elimination}
on any square matrix, while it has been specifically optimized for any 24x24 square matrix.
It is optimized to yield the most efficient execution, whilst keeping the price as low as
possible. Therefore the performance has often been compromised in favour of cheaper components.

The software will reduce the matrix to {\it upper triangle form}, which is characterized
as a matrix with it's diagonal containing only ones, and all elements below the diagonal
containing only zeros. The second step in the elimination process, {\it back substitution},
is not handled by this system.

\section{Optimization}
Both software and hardware have been in consideration when optimizing the system. The hardware
is heavily dependent on how the program is executed, therefore the program have been 
rewritten in such way that it would be beneficial for an optimal hardware configuration.

\subsection{Software}
The algorithm iterates through the matrix in several different loops. The outermost loop iterates along
the main diagonal ({\it the pivot elements}) of the matrix. In each loop, we do the two following
operations; we minimize the current row by dividing each element with the pivot element, and then we
subtract each underlying row with the previously minimized row multiplied with the respective
pivot elements. By the end of the first outmost loop, we end up with a matrix with the first pivot
element as a one, and all the elements below it as zeros, with the first row in its final form.
Then successively after each row being computed, we will end up with our result matrix.

The first approach for writing the program was to directly translate an elimination algorithm written in C-code to Assembler,
and not bother with any optimizations. After confirming that we had a working program, we started
find places in the code where we could do something to speed up the performance. One optimization we
made in the beginning, was to convert the use of arrays into pointers instead. With arrays, you have
to calculate each new address for every iteration of the loop. This will increase the number of
instructions executed in each loop, especially in the most deep-nested loops. In the algorithm we
use, we have a loop nested inside two other loops, which will add up to a lot of clock cycles in the
end. By using pointers instead, we can omit the use of indices of the arrays, and instead only
increment the address by one byte.

After running the program with the provided PET-tool, we could analyze if we had any {\it branch
hazards} or {\it load-use hazards}. Branch hazards occur when an instruction calculates a value
for a conditional branch, that is directly subsequent. The CPU will then have to stall one clock
cycle due to the delay in the pipeline, which is a waste of performance. By reordering the
instructions, for example inserting a completely independent instruction in between, we can avoid
this type of hazard. Load-use hazards work in a similar way; when loading a value to a register,
then immediately using it in another instruction, the CPU will have to stall one clock cycle.

\subsection{Hardware}
When we were choosing the components we had to consider several things. Since the goal of this project was to get the best
performance for as cheap as possible, we could not simply try to get the fastest execution time possible. With this in mind,
we tested several different configurations. In the following subsections we describe our thought process when deciding
which hardware configuration to use.

\subsubsection{Memory}
Since the entire 24x24 matrix is stored in memory we have to read memory every time we need a new value from the matrix,
and write to memory every time we want to update a value in the matrix. This adds up to a lot of memory accesses, so a 
fast memory is going to greatly improve the execution time. 

\subsubsection{I-Cache}
There are not many instructions in the loops in our program. A large I-Cache is not necessary to store the instructions of each loop
while it is running, which means that a larger cache is not going to be a big enough performance increase to make the price increase
worth it.

\subsubsection{D-Cache}
We had to hit a balance between price and performance here. Larger cache gives better execution times, 
since there is such a large amount of memory accesses, but the price increase is steeper than the performance increase. 


\subsection{Results}
H�r l�gger vi in alla testresultat och visar vilken som �r b�st etc.

After trying a lot of different configuration, this was the best one we could find:

\begin{table}
\begin{tabular}{l|l|l|}
{\bf Configuration} \\ \hline
I-Cache size(words) & 32 \\ \hline
I-Cache associativity & 1 \\ \hline
I-Cache block size(words) & 4 \\ \hline
D-Cache size(words) & 64 \\ \hline
D-Cache associativity & 2 \\ \hline
D-Cache block size(words) & 4 \\ \hline
Processor frequency(MHz) & 450 \\ \hline
Memory access time(ns, first/other) &  30 / 6\\ \hline
Write buffer Size(words) & 4 \\ \hline
{\bf Results} & \\ \hline
Clock cycles & 140 314 \\ \hline
Execution time(\mu s) & 311.809 \\ \hline
Total component cost & 3.37 \\ \hline
Price x performance(\mu sC\$) & 1050.796\\ \hline
\end{tabular}
\end{table}









\begin{table}   
\centering
\scriptsize
\def\arraystretch{1.1}              
\begin{tabularx}{14cm}{X|cccccc|cccccc}
              & \multicolumn{6}{c|}{\bf Gothenburg}                                   & \multicolumn{6}{c}{\bf Stockholm}                                    \\ \hline
{\bf Year}    & {\bf Jan} & {\bf Feb} & {\bf Mar} & {\bf Apr} & {\bf May} & {\bf Jun} & {\bf Jan} & {\bf Feb} & {\bf Mar} & {\bf Apr} & {\bf May} & {\bf Jun} \\ \hline
{\bf 2002}    & 153       & 95        & 48        & 27        & 71        & 116       & 65        & 42        & 28        & 8         & 34        & 94        \\
{\bf 2003}    & 71        & 34        & 21        & 96        & 77        & 64        & 19        & 21        & 2         & 44        & 58        & 44        \\
{\bf 2004}    & 65        & 38        & 61        & 36        & 27        & 83        & 37        & 29        & 25        & 13        & 47        & 108       \\
{\bf 2005}    & 126       & 43        & 41        & 25        & 39        & 73        & 44        & 38        & 14        & 11        & 56        & 72        \\
{\bf 2006}    & 66        & 41        & 52        & 94        & 71        & 42        & 10        & 29        & 28        & 28        & 46        & 32        \\
{\bf 2007}    & 159       & 28        & 81        & 40        & 57        & 130       & 73        & 26        & 19        & 17        & 25        & 67        \\
{\bf 2008}    & 166       & 89        & 108       & 33        & 11        & 67        & 69        & 23        & 41        & 28        & 17        & 56        \\
{\bf 2009}    & 57        & 31        & 48        & 11        & 80        & 58        & 21        & 24        & 36        & 5         & 25        & 80        \\
{\bf 2010}    & 30        & 48        & 46        & 32        & 31        & 53        & 26        & 29        & 22        & 23        & 31        & 33        \\
{\bf 2011}    & 72        & 65        & 40        & 34        & 67        & 58        & 33        & 38        & 11        & 10        & 34        & 57        \\
{\bf 2012}    & 88        & 55        & 9         & 75        & 74        & 95        & 42        & 45        & 18        & 62        & 20        & 160       \\
{\bf 2013}    & 44        & 17        & 3         & 46        & 73        & 139       & 29        & 30        & 3         & 27        & 17        & 46        \\
{\bf 2014}    & 66        & 107       & 32        & 47        & 81        & 52        & 39        & 35        & 43        & 25        & 35        & 28        \\ \hline
{\bf Average} & 89        & 53        & 45        & 46        & 58        & 79        & 39        & 31        & 22        & 23        & 34        & 67        \\ \hline
{\bf Normal value \newline
(1961-1990)}  & 61        & 40        & 49        & 41        & 49        & 59        & 39        & 27        & 26        & 30        & 30        & 45 \\
\end{tabularx}
\vspace{0.5cm}

\begin{tabularx}{14cm}{X|cccccc|cccccc}
{\bf Year}    & {\bf Jul} & {\bf Aug} & {\bf Sep} & {\bf Oct} & {\bf Nov} & {\bf Dec} & {\bf Jul} & {\bf Aug} & {\bf Sep} & {\bf Oct} & {\bf Nov} & {\bf Dec} \\ \hline
{\bf 2002}    & 101       & 35        & 17        & 86        & 82        & 16        & 114       & 4         & 13        & 45        & 32        & 9         \\
{\bf 2003}    & 143       & 32        & 25        & 101       & 76        & 119       & 75        & 53        & 20        & 32        & 43        & 78        \\
{\bf 2004}    & 98        & 128       & 100       & 115       & 82        & 76        & 55        & 49        & 39        & 45        & 47        & 40        \\
{\bf 2005}    & 107       & 83        & 60        & 75        & 93        & 53        & 122       & 65        & 11        & 28        & 44        & 54        \\
{\bf 2006}    & 78        & 190       & 55        & 202       & 179       & 193       & 33        & 146       & 23        & 98        & 43        & 34        \\
{\bf 2007}    & 129       & 56        & 121       & 47        & 82        & 108       & 51        & 17        & 74        & 39        & 48        & 45        \\
{\bf 2008}    & 57        & 167       & 81        & 142       & 100       & 37        & 42        & 154       & 27        & 64        & 63        & 50        \\
{\bf 2009}    & 170       & 79        & 56        & 84        & 128       & 53        & 89        & 54        & 32        & 68        & 52        & 45        \\ 
{\bf 2010}    & 159       & 118       & 74        & 130       & 80        & 48        & 74        & 105       & 73        & 34        & 58        & 45        \\
{\bf 2011}    & 121       & 156       & 129       & 107       & 39        & 146       & 13        & 90        & 60        & 44        & 14        & 75        \\
{\bf 2012}    & 97        & 88        & 146       & 146       & 92        & 78        & 61        & 121       & 58        & 72        & 56        & 66        \\
{\bf 2013}    & 51        & 49        & 56        & 106       & 72        & 154       & 43        & 55        & 39        & 58        & 61        & 51        \\
{\bf 2014}    & 39        & 142       & 36        &           &           &           & 44        & 115       & 83        &           &           &           \\ \hline
{\bf Average} & 104       & 102       & 74        & 103       & 85        & 83        & 63        & 79        & 42        & 48        & 43        & 46        \\ \hline
{\bf Normal value \newline
(1961-1990)}  & 68        & 75        & 80       & 83         & 82        & 72        & 72        & 66        & 55        & 50        & 53        & 46        \\
\end{tabularx}
\caption{Precipitation, mm/month}
\label{tab:gbg/sthlm}
\end{table}


\begin{thebibliography}{3}
\bibitem{smhi}     Monthly observations of precipitation 2002-2014, SMHI.                                \\
                   \url{http://data.smhi.se/met/climate/time_series/month/vov_pdf/}, \mbox{visited} 2014-10-12.
\bibitem{wmo}      Climate Data and Data Related Products, WMO.                                                     \\
                   \url{http://www.wmo.int/pages/themes/climate/climate_data_and_products.php}, \mbox{visited} 2014-10-11.
\bibitem{critical} Student t-Value Calculator, Dr. Daniel Soper, 2014.                                      \\
                   \url{http://www.danielsoper.com/statcalc3/calc.aspx?id=10},
                   \mbox{visited} 2014-10-12.
\end{thebibliography}
\end{document}